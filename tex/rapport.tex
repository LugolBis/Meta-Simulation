\documentclass{article}

\title{Rapport IN620}
\date{\today}
\author{GHARIB ALI BARURA Sama, DESMARES Loïc}

\begin{document}
  \maketitle

  \newpage

  \section{Question 1}
    Pour représenter un automate cellulaire, il suffit d'en enregistrer la fonction de
  transition. Pour cela, on utilisera un tableau unidimensionel de taille 8 de manière
  à pouvoir obtenir le nouvel état d'une cellule \(c_i\) à partir du
  \((c_{i-1} c_i c_{i+1})_2\)ième élément de ce tableau.

  \section{Question 2}
      Une configuration d'un automate cellulaire à l'instant t n'a besoin que des données
  de sa bande. On représentera un état de cellule par une énumération `Vivant, Mort, Blanc`.
  Une cellule est un 2-uplet d'état représentant l'état courant et l'état au temps t+1.
  Pour représenter la bande (grille) en elle-même, le mieux est d'utiliser une liste doublement
  chaînée. Cela permet de garantire les opérations d'ajout à gauche et à droite en un temps
  O(1). De plus, tout indexage de la bande se fait au cours d'une itération complète, ainsi
  le problème de la complexité O(n) pour accéder à un élément précis ne se pose pas.


  
\end{document}

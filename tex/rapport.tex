\documentclass{article}

\title{Rapport IN620}
\date{\today}
\author{GHARIB ALI BARURA Sama, DESMARES Loïc}

\begin{document}
  \maketitle

  \newpage

  \section{Question 1}
    Pour représenter un automate cellulaire, il suffit d'en enregistrer la fonction de
  transition. Pour cela, on utilisera un tableau unidimensionel de taille 8 de manière
  à pouvoir obtenir le nouvel état d'une cellule \(c_i\) à partir du
  \((c_{i-1} c_i c_{i+1})_2\)ième élément de ce tableau.

  \section{Question 2}
      Une configuration d'un automate cellulaire à l'instant t n'a besoin que des données
  de sa bande. On représentera un état de cellule par une énumération `Vivant, Mort`.
  Une cellule est un 2-uplet d'état représentant l'état courant et l'état au temps t+1.
  Pour représenter la bande (grille) en elle-même, le mieux est d'utiliser une liste doublement
  chaînée. Cela permet de garantir les opérations d'ajout à gauche et à droite en un temps
  O(1). De plus, tout indexage de la bande se fait au cours d'une itération complète, ainsi
  le problème de la complexité O(n) pour accéder à un élément précis ne se pose pas.

  \section{Question 14}
      Soit le problème HALTING-CELLULAR-AUTOMATON : étant donné < A > le code
  d’un automate cellulaire, s ∈ S et un mot w appartenant à S * , décider si A sur l’entrée w va avoir une configuration
  contenant s lors de son calcul.

  Au travers de la question 13 nous avons démontrer (et implémenter) une solution permettant de traduire un 
  automate cellulaire en une machine de turing.
  Nous admettons donc que nous pouvons aisément passer d'un automate cellulaire à une machine de turing.

  Démonstration par l'absurde
  Supposons que Lhca le language reconnaissant HALTING-CELLULAR-AUTOMATON est décidable tel que il existe 
  une machine de turing HCA qui décide Lhca.
  On définit H une machine de turing qui décide HALT :
  - Entrée : (<M>, w)
  - Si HCA accepte (<M>,w) alors ACCEPT
    Sinon ACCEPT

  . Si (<M>, w) appartient à HALT alors HCA s'arrête sur w. On a HCA qui calcule si une configuration donnée 
  appartient aux configurations obtenues lors du calcul sur le mot w.
  . Si (<M>,w) n'appartient pas à HALT alors HCA ne s'arrête pas sur w. Or HCA est décidable donc finit 
  nécessairement par s'arrêter et ACCEPT/REJECT, donc HALT accepte l'entrée (<M>, w).

  Donc HALT décidable : contradiction, HALTING-CELLULAR-AUTOMATON est indécidable.
  
\end{document}
